%
%                       This is a basic LeTeX Template
%                       for the First Year Summary 
\documentclass[a4paper,12pt]{article}
\usepackage{head,fullpage}    % Add local fullpage macros
\usepackage[pdftex]{graphicx} % Graphicx package with pdf support
\parindent=0pt          %  Switch off indent of paragraphs 
\parskip=5pt            %  Put 5pt between each paragraph 
\footruleheight{1pt}
\headruleheight{1pt}
\lfoot{\small School of Physics and Astronomy}
\lhead{First Year Summary}
\rhead{- \thepage}
\cfoot{}
\rfoot{Date: 18-5-09} 
%
%                       This section generates a title page
%                       Edit only the sections indicated to put
%                       in the project title, your name, supervisor,
%                       project length in weeks and submission date
%
\begin{document}
\begin{minipage}[b]{110mm}
        {\Huge\bf School of Physics \\and Astronomy
        \vspace*{17mm}}
\end{minipage}
\hfill
\begin{minipage}[t]{40mm}               
        \makebox[40mm]{
        \includegraphics[width=40mm]{crest}}
\end{minipage}
\par\noindent                                           % Centre Title, and name
\vspace*{5mm}
\begin{center}
        \Large\bf Proposed Project Title\\
        \Large\bf First Year Summary
\end{center}
\vspace*{3mm}
\begin{center}
        \bf Patrick Sinclair\\                 % Replace with your name
        June 2009                         % Submission Date
\end{center}
\vspace*{5mm}
{\bf Supervisor:} Dr. A.N. Other                % Change to suit

%                                               Through page and setup 
\section{Project Outline}

Outline of your project, covering the scientific background, the rational
for your intended area of study and {\bf your progress to date}. 
This should concentrate on why this subject
area is important, interesting, worth studying and where you hope/believe your
work will lead. You should also describe the context of your PhD � how does it fit into the international agenda?

This section should be 2-3 pages and written for the {\em general scientific
reader}, (the typical reader of New Scientist, and in particular
should {\it not} include any mathematical derivations.)

The whole report {\em must not} exceed {\bf four} pages.

\section{Critical Dependencies}

If there are {\it critical dependencies} in your research work, detail them
here, for example 
``{\it\ldots my experimental work depend on the LHC being fixed\ldots}'' 
and what steps have been taken to deal(?) with them.

Also in this section outline any concerns about your progress, and 
in particular any aspect that you feel are impeding your progress.
  
\section{Training and Courses}

In this section list the courses you {\em have} attended this year
including their origin (eg: SUPA / UG / Summer School), whether this
course was assessed, and the number of contact hours, here is a typical
simple table.

{\bf Course Attended in First Year}
\begin{center}
\begin{tabular}{l|c|c|c}
  Name    &     Origin    & Assessed   & Hours\\
  \hline
  RQFT    &     SUPA      & Yes        & 18\\
  Biophysics & UG         & No         & 15\\
  Unix Skills & Transkills & No        & 3\\
\end{tabular}
\end{center}

Include a {\em provisional} list of courses/training that you are
going to attend {\em next} year, 

{\bf Courses to be Attended in 2009-2010}
\begin{center}
\begin{tabular}{l|c|c}
  Name    &     Origin    & Assessed \\
  \hline
  Standard Model     &     SUPA      & Yes\\
  Advanced Cosmology & UG         & No\\
  Thesis preparation & Transkills & No\\
\end{tabular}
\end{center}

Also feel free to suggest any changes, additional courses/training that you
feel would have been valuable.

\section{Teaching}

In this section list the taught undergraduate teaching you have undertaken.
\begin{center}
\begin{tabular}{l|c|c}
  Name     &    Semester   & Hours\\
\hline
  Physics1A Workshops & 1 & 12\\
\end{tabular}
\end{center}
also whether you intend to increase or reduce your teaching commitment 
next year.
 

\end{document}

